\subsubsection{Software}
The software we have made use of on the lab machine is.
\begin{itemize}
  \item \textbf{CUDA Toolkit 4.2} which provides a comprehensive development environment for C and C++ developers building GPU-accelerated applications.
  \item \textbf{NVIDIA CUDA Compiler (NVCC)} which deals with the compilation and transfer of CUDA code. The code is divided into CPU-bound and GPU-bound code, delegating the compilation of the former to a C/C++ compiler, and compiling the latter. Finally, the compiled binaries are transferred to their respective destinations.
  \item \textbf{CUDA Command Line Profiler} for measuring GPU performance at a higher detail in order to reason about results.
  \item \textbf{Python} for writing scripts to analyse both profiled results and benchmark scores, performing database operations, and assist in running commands remotely.
  \item \textbf{Bash} for writing scripts to perform database operations, and automating experiment runs.
  \item \textbf{cuda-gdb} for debugging CUDA code.
  \item \textbf{cuda-memcheck} is a good complement to cuda-gdb, and is particularly useful for diagnosing illegal memory accesses.
\end{itemize}